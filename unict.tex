% !TeX root = unict.tex


\input{cmd}

\pagenumbering{arabic}

\begin{document}


% QUI È POSSIBILE SCEGLIERE I MARGINI
\newgeometry{top=3cm, bottom=4cm}

\thispagestyle{empty}

% QUI È POSSIBILE INSERIRE UNIVERSITÀ, DIPARTIMENTO E CORSO DI LAUREA
\universita
{UNIVERSITÀ DEGLI STUDI DI CATANIA}
{DIPARTIMENTO DI SCIENZE CHIMICHE}
{CORSO DI LAUREA IN CHIMICA}

% QUI È POSSIBILE INSERIRE IL NOME DELLO STUDENTE

\studente{Francesco MESSINA}

% QUI È POSSIBILE INSIERIRE IL TITOLO DELLA TESI

\ttitle{Il titolo della tesi}

\tipo{ELABORATO FINALE}

\fskip\fskip

% QUI È POSSIBILE INSIERIRE IL NOME DEL RELATORE

\relatore{Chiar.mo Prof. Andrea Pappalardo}

% Nel caso sia presente anche un correlatore è possibile aggiungere
% \correlatore{NOME CORRELATORE}

\fskip

% QUI È POSSIBILE INSERIRE LA DATA DI LAUREA
\data{Novembre (si spera) 2024}

\pagenumbering{roman}

% QUI È POSSIBILE INSERIRE UNA DEDICA

\dedica{El Psy Kongroo}

% INDICE DEI CONTENUTI

\tableofcontents

% INDICE DELLE TABELLE

\listoftables

% INDICE DELLE IMMAGINI

\listoffigures

\cleardoublepage

\pagenumbering{arabic}

% QUI È POSSIBILE INSERIRE INTRODUZIONE, CAPITOLI E CONLCUSIONI

\chapter{Introduzione}
\input{chapters/cap1}

\chapter{Titolo del capitolo 1}
\input{chapters/cap1}

\chapter{Titolo del capitolo 2}
\input{chapters/cap2}

\chapter{Conclusioni}
\input{chapters/conclusioni}

% QUI È POSSIBILE INSERIRE LA BIBLIOGRAFIA

\input{chapters/bibliografia}













\end{document}